%! Author = raquelmolire
%! Date = 1/10/2022

% Document
En este apartado se incluye la situación actual del uso de técnicas de DL para el diagnóstico del Alzheimer, es decir,
la información sobre las últimas actualizaciones e investigaciones ya disponibles al público sobre este tema.

En la última década, las técnicas de aprendizaje profundo han alcanzado una enorme popularidad en el ámbito del análisis
de imágenes médicas y como herramienta de diagnóstico y seguimiento de enfermedades.
Al mismo tiempo, ha incrementado el uso de técnicas de neuroimagen para el diagnóstico de la EA, por lo que la unión de
ambas técnicas, de aprendizaje profundo y de neuroimagen, ha demostrado ser una combinación ideal para realizar un
correcto diagnóstico de la enfermedad y poder predecir la evolución de la misma.

El número de artículos de investigación de la detección de la EA que se han publicado en los últimos años ha crecido
exponencialmente.
Recogiendo información sobre múltiples resultados con el uso de distintos biomarcadores o distintos modelos profundos.

La mayoría de los artículos publicados tratan el tema de la misma manera: planteando un nuevo modelo de aprendizaje
profundo y no llegando a explorar mejoras a otras alternativas o reflexionar sobre qué recursos son mejores, siendo el
resultado un cúmulo de estudios que no producen una conclusión concreta a qué métodos son más óptimos y por qué.
Existe un artículo de la universidad en Newcastle, Australia, que realiza una revisión de la literatura de más de 100
artículos y que detalla los hallazgos y tendencias, examinando biomarcadores y características útiles, técnicas de
preprocesamiento necesarias y diferentes métodos de tratamiento de neuroimágenes.
Este estudio se ha utilizado de referencia para este apartado.

\subsection{Biomarcadores que intervienen en la detección de la EA}\label{subsec:biomarcadores-estado-del-arte}
Para la detección de la EA las técnicas de neuroimagen no invasivas más utilizadas en los estudios son: MRI, fMRI
(Imagen por resonancia magnética funcional)  y PET.

La fMRI es un tipo especial de MRI que proporciona un conjunto de imágenes del flujo sanguíneo de ciertas partes del
cerebro y también son utilizadas para evaluar daños cerebrales.

De estas técnicas, MRI es el biomarcador más utilizado en la literatura debido a que ha demostrado un alto rendimiento.
A pesar de que varios estudios han demostrado que la MRI es más discriminatoria en comparación con la PET, otros estiman
que la MRI es igual de discriminatoria que la PET o ligeramente menos.

Además MRI es la técnica más frecuente en las pruebas médicas para la detección de la EA ya que proporciona información
precisa y el coste de su realización es menor.
Lo que sitúa a esta técnica como la mejor candidata para avanzar en la investigación del diagnóstico de la enfermedad.
