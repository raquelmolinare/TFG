%! Author = raquelmolire
%! Date = 24/09/2022

% Document

A día de hoy, gracias a los avances en ciencias de la computación, la inteligencia artificial (IA) se está convirtiendo
en una parte fundamental de la atención médica actual.
Algoritmos y aplicaciones impulsadas por IA se usan en el día a día para ayudar a los profesionales en entornos clínicos
y en el ámbito de la investigación.

La enfermedad del Alzheimer (EA)  es una enfermedad neurodegenerativa que afecta a aproximadamente 1.200.000 personas
en España y a más de 50 millones de personas en el mundo.
 Es una enfermedad progresiva , o lo que es lo mismo, que empeora con el tiempo y para la que no existe una cura, solo
la posibilidad de aplicar tratamientos que ralenticen su avance.
Para que estos tratamientos no resulten perjudiciales para la salud del paciente deben realizarse en las primeras etapas
de la enfermedad.
Por ello un diagnóstico temprano puede ser determinante para el paciente, pero, en la mayoría de casos, detectar esta
enfermedad en las primeras fases de la misma es una tarea muy compleja.

Investigaciones previas han concluido que las primeras lesiones cerebrales pueden aparecer incluso 20 años antes de que
aparezcan los primeros síntomas del Alzheimer.
De hecho una de las evaluaciones médicas para la detección de la enfermedad se basa en el diagnóstico por imágenes
cerebrales mediante la realización de un examen imagenológico, que puede incluir técnicas como: Imagen por Resonancia
Magnética (MRI), Tomografía Computarizada (TC) o Tomografía por emisión de positrones (PET).

A día de hoy el aprendizaje profundo o deep learning (DL)  ha levantado mucho interés en el mundo de la medicina.
El Deep learning es un subconjunto del machine learning (ML) en IA, que simula el comportamiento del cerebro humano en
el procesamiento de datos y el reconocimiento de patrones para resolver problemas complejos de toma de decisiones.

Es por esto que el uso de DL puede ayudar a un diagnóstico temprano, ya que estos sistemas pueden ser utilizados para la
detección de anomalías en imágenes médicas donde destaca la rapidez de la detección, sirviendo como herramienta de
prevención y seguimiento de la enfermedad.

También cabe destacar que la enfermedad del Alzheimer presenta diferentes fases y la velocidad a la que avanza la
enfermedad por las diferentes fases varía, por lo que es más difícil realizar predicciones a largo plazo.
Se usan escalas con distinto número de fases:
\begin{itemize}
    \item La escala de deterioro global (GDS) se divide en siete fases que dependen del valor del deterioro cognitivo y
    más común utilizarla para escalar la demencia senil.
    Sus fases son: ausencia de alteración cognitiva, disminución cognitiva muy leve, defecto cognitivo leve, defecto
    cognitivo moderado, defecto cognitivo moderado-grave, defecto cognitivo grave y defecto cognitivo muy grave.
    \item La escala de clasificación de la demencia clínica (CDR)  se divide en cinco fases, es la más utilizada en el
    área de investigación y evalúa diferentes parámetros como la memoria, la orientación, la resolución de problemas,
    el juicio, etc.
    Sus fases son: Cognitivamente sano (CDR 0), demencia cuestionable (CDR 0.5), demencia leve (CDR 1), demencia
    moderada (CDR 2) y demencia grave (CDR 3).
    \item La escala de clasificación común solo tiene en cuenta tres fases, es la más usada en la comunicación
    médico-familia, ya que es la más sencilla de comprender.
    Las fases que tiene en cuenta son: Enfermedad de Alzheimer leve (etapa temprana), enfermedad de Alzheimer moderada
    (etapa media) y enfermedad de Alzheimer grave (etapa final).\\
\end{itemize}

A día de hoy, realizar estudios en esta área significa conseguir grandes avances en el mundo de la medicina y del
diagnóstico del Alzheimer, ayudando a mejorar la vida de los pacientes y las familias.
Por lo tanto, la motivación de este trabajo se centra en el estudio del uso de técnicas de Deep Learning para el
diagnóstico del Alzheimer a partir de imágenes cerebrales.