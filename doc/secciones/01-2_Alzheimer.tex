%! Author = raquelmolire
%! Date = 25/09/2022

El alzheimer es una enfermedad neurodegenerativa, progresiva, irreversible y terminal en la que se produce una pérdida
de neuronas principalmente relacionada con dos tipos de alteraciones cerebrales:  la acumulación anormal de placas
seniles de proteína beta-amiloide y ovillos neurofibrilares de proteína Tau.

\subsection{Epidemiología de la enfermedad}\label{subsec:epidemiologia}
Es la forma más común de demencia, una pérdida de la función cerebral que afecta la memoria, el pensamiento, el
lenguaje, el juicio y el comportamiento.
Se estima que entre un 60 y un 80 por ciento de los casos de demencia se producen a causa de la EA en los países
desarrollados.

Esta patología tiene una mayor frecuencia en personas mayores de 65 años, siendo la prevalencia de un 7\% en este grupo
de población, y aproximándose al 50\% en mayores de 85 años.
Aunque en otros casos más extremos y mucho menos frecuentes puede ser desarrollada a partir de los 30 años, siendo
denominada en este caso como Enfermedad de Alzheimer precoz o de aparición temprana.
En término medio, una persona con Alzheimer vive de 4 a 8 años después de ser diagnosticada, pero puede vivir hasta 20
años dependiendo de otros factores como es, por ejemplo, la etapa en la que se diagnostique la enfermedad.

Actualmente, en España la cifra de personas afectadas por la enfermedad del Alzheimer es de aproximadamente 1.200.000,
aproximándose a las 5.000.000 personas si contamos con la familia.
