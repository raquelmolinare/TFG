\chapter{ADNI}\label{ch:aped.a}
\textbf{Alzheimer’s Disease Neuroimaging Initiative (ADNI)\footnote{\href{https://adni.loni.usc.edu/}{ADNI}}} es un
estudio longitudinal multicéntrico en el que se desarrollan biomarcadores genéticos, bioquímicos y de imagen para la
detección temprana y el seguimiento de la enfermedad de Alzheimer, y que tiene como objetivos principales:
\begin{itemize}
    \item La detección de la EA en la fase más temprana posible y determinar procedimientos para hacer un seguimiento
    de la enfermedad con biomarcadores.
    \item Apoyar los avances en la intervención, prevención y tratamiento de la EA mediante el uso de nuevos métodos de
    diagnóstico en las etapas más tempranas posible, actuando así en el momento en el que la intervención puede ser más eficaz.
    \item Contribuir en la investigación de la EA, proporcionando los datos entre investigadores de todo el mundo.\\
\end{itemize}

Los datos utilizados en este TFG se han obtenido de esta base de datos.
Su acceso es gratuito, pero es necesario realizar una solicitud en la que se indica el motivo por el que se requiere el
acceso a los datos de ADNI, la afiliación institucional a la que se pertenece como investigador, y posteriormente
obtener una aprobación de la solicitud.
Además implícitamente se acepta un acuerdo de uso de datos, ya que estos datos no pueden ser utilizados con fines comerciales.

Al solicitar el acceso se puede hacer la solicitud a tres estudios diferentes: ADNI,
AIBL\footnote{\href{https://aibl.csiro.au/}{AIBL}} y DOD-ADNI. Cada estudio tiene su propio acuerdo de uso de datos.
En cuanto al mantenimiento de la cuenta, se requiere de la presentación de una actualización anual, en caso contrario,
la cuenta expirará automáticamente (se envía previamente un recordatorio por correo electrónico).

ADNI comenzó en 2004 bajo la dirección del \textit{Dr. Michael W. Weiner}, obteniendo financiación para un estudio
inicial de 5 años: ADNI-1 con el objetivo de desarrollar biomarcadores como medidas de resultado para ensayos clínicos,
el cual se amplió varias veces, la primera en 2009 dando lugar a ADNI-GO (Grand Opportunities) con el propósito de
examinar los biomarcadores en las primeras fases de la enfermedad, y la segunda y tercera ampliación en 2011 y 2016 con
ADNI-2 y ADNI-3 para el desarrollo de biomarcadores como predictores del deterioro cognitivo y para el estudio del uso
de PET de tau y técnicas de imagen funcional en ensayos técnicos respectivamente.

ADNI incluye participantes de entre 50 y 90 años de edad de Estados Unidos y Canadá, que se someten a una serie de
pruebas iniciales que se repiten en intervalos en los posteriores cuatro años.
El mayor número de participantes se encuentra en la franja de 70 a 79 años.
Se dividen en distintos grupos de investigación según el grado de enfermedad de Alzheimer presente,desde individuos
sanos, pacientes con deterioro cognitivo leve hasta pacientes que padecen EA.

Los datos de ADNI están administrados por LONI\footnote{\href{https://loni.usc.edu/}{LONI}} a partir de
IDA\footnote{\href{https://ida.loni.usc.edu/}{IDA}} que es un recurso online seguro para archivar y compartir datos
sobre neurociencia.

IDA (Image and Data Archive) está gestionado por el Laboratorio de Neuroimagen, del inglés Laboratory Of Neuro Imaging
(LONI), del Instituto de Neuroimagen e Informática Mark y Mary Stevens de la
USC\footnote{\href{https://www.ini.usc.edu/}{USC}}.
Este laboratorio gestiona datos de neuroimagen para estudios de investigación multicéntricos desde finales de los
años 90 y como nexo de unión entre estos estudios y LONI se encuentra IDA, que proporciona, a partir de una
infraestructura robusta y fiable, herramientas y recursos para buscar, visualizar y compartir una amplia gama de datos
neurocientíficos y facilita la colaboración entre científicos de todo el mundo.
