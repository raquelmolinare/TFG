\thispagestyle{empty}

\begin{center}
{\large\bfseries \myTitle}\\
\end{center}
\begin{center}
       \myName\\
\end{center}

\vspace{0.7cm}
\noindent{\textbf{Palabras clave}: Alzheimer , Deep Learning, Clasificación, Imagen por resonancia magnética,
       Biomarcadores, Neuroimagen, Axial, Coronal, Sagital}\\

\vspace{0.7cm}
\noindent{\textbf{Resumen}}\\
El Alzheimer es la forma más común de demencia, afecta a millones de personas a día de hoy y no tiene cura.
Un diagnóstico en las primeras etapas de la enfermedad puede ser determinante para la vida del paciente.
Los avances en ciencias de la computación han supuesto grandes logros en el mundo de la medicina.
Actualmente, la investigación en el ámbito de técnicas de Deep Learning para el diagnóstico de la enfermedad puede
resultar de gran ayuda para mejorar su predicción y seguimiento, mejorando notablemente la calidad de vida de las
personas que la sufren.
En este proyecto se realiza una revisión y estudio del uso de este tipo de técnicas.
El objetivo del proyecto es evaluar qué plano cerebral, axial, coronal o sagital, ofrece mejor rendimiento para la
clasificación de la enfermedad de Alzheimer usando técnicas de Deep Learning y comprobar si se puede facilitar su uso a
partir de una aplicación web.
\cleardoublepage
\thispagestyle{empty}


\begin{center}
{\large\bfseries \myTitleEn}\\
\end{center}
\begin{center}
       \myName\\
\end{center}

\vspace{0.7cm}
\noindent{\textbf{Keywords}: Alzheimer's disease, Deep Learning, Classification, Magnetic Resonance Imaging, Biomarkers,
       Neuroimaging, Axial, Coronal, Sagittal}\\

\vspace{0.7cm}
\noindent{\textbf{Abstract}}\\

Alzheimer's disease is the most common form of dementia, millions of people suffer from it and it has no cure.
A diagnosis in the early stages of the disease can be decisive for the patient's life.
Advances in computer science have made great achievements in the world of medicine.
Currently, the research in the area of Deep Learning techniques for the diagnosis of Alzheimer's disease can be great
to improve its prediction and monitoring.
A review and study of the use of this type of techniques is done in this project.
The goal of the project is to evaluate which brain plane, axial, coronal or sagittal, offers better results for the
classification of Alzheimer's disease using Deep Learning techniques and to check if its use can be easier through a
web application.

%\cleardoublepage
%\thispagestyle{empty}
%
%\noindent\rule[-1ex]{\textwidth}{2pt}\\[4.5ex]
%
%Yo, \textbf{Raquel Molina Reche}, alumna de la titulación Ingeniería Informática de la \textbf{Escuela Técnica Superior
%de Ingenierías Informática y de Telecomunicación de la Universidad de Granada}, con DNI 49627634M, autorizo la
%ubicación de la siguiente copia de mi Trabajo Fin de Grado en la biblioteca del centro para que pueda ser
%consultada por las personas que lo deseen.
%
%\vspace{6cm}
%
%\noindent Fdo: Raquel Molina Reche
%
%\vspace{2cm}
%
%\begin{flushright}
%Granada a X de mes de 201 .
%\end{flushright}


\cleardoublepage
\thispagestyle{empty}

\noindent\rule[-1ex]{\textwidth}{2pt}\\[4.5ex]

Dª. \textbf{\myProf}, Profesora del \myDepartment de la \myUni.

\vspace{0.5cm}

\textbf{Informa:}

\vspace{0.5cm}

Que el presente trabajo, titulado \textit{\textbf{\myTitle}},
ha sido realizado bajo su supervisión por \textbf{\myName}, y autorizo la defensa de dicho trabajo ante el tribunal
que corresponda.

\vspace{0.5cm}

Y para que conste, expido y firmo el presente informe en \myLocation a \myTime.

\vspace{1cm}

\textbf{La directora:}

\vspace{5cm}

\noindent \textbf{\myProf}

\chapter*{Agradecimientos}
\thispagestyle{empty}

       \vspace{1cm}

En primer lugar a mis padres, por todo lo que han trabajado para darme lo mejor, por haber sido mi motor en todo momento,
y que junto a mis hermanas han sido un apoyo fundamental durante estos años. \\

A mis abuelos, en especial a mi abuelo Manuel, diagnosticado con Alzheimer desde hace 8 años y razón principal de este
TFG. \\

Y a Rosa, mi tutora, por guiarme durante este camino y haber hecho posible este proyecto. \\

