\chapter{Conclusiones y Trabajos Futuros}\label{ch:conclusiones-y-trabajos-futuros}

\section{Conclusiones}\label{sec:conclusiones}
Con este proyecto se pretendía realizar un estudio de técnicas de DL para el diagnóstico de la EA.
En concreto se centró el objetivo general del estudio en identificar con qué plano cerebral se obtiene mejor rendimiento
a partir de biomarcadores de tipo MRI. En la literatura actual no quedaba bien definido qué plano era mejor.

Tras realizar un análisis de la base de datos se observó que el conjunto de datos disponible era limitado y no estaba
balanceado por lo que se ha balanceado y realizado la extracción de imágenes de 2D de los biomarcadores MRI de 3D,
generando los datasets de entrenamiento correspondientes a los casos de análisis.
Añadiendo también una comparación entre la clasificación de imágenes usando 2 clases, CN y AD, y usando 3 clases, CN,
MCI y AD, de manera que se realiza un estudio más en profundidad de estas técnicas para la medición del deterioro
cognitivo.

Los resultados muestran que una clasificación mediante 2 clases proporciona mucho mejor rendimiento que la clasificación
mediante 3 clases, sucediendo con esta tendencia para todas las vistas cerebrales y que se debe fundamentalmente a que
el conjunto de datos es reducido.

En cuanto a la pregunta final, sobre qué plano es mejor, al utilizarse el mismo conjunto de biomarcadores para la
extracción de los cortes 2D, queda demostrado que el plano coronal es el mejor de los 3, con un rendimiento del 72\%
frente al 69\% del plano axial y el 53\% del plano sagital.

Además, teniendo en cuenta que el plano coronal engloba las tres regiones más importantes del cerebro relacionadas con
la EA como son el hipocampo, la corteza y los ventrículos.
Se concluye que los resultados son acordes a lo esperado.
Pudiendo servir como guía para futuras conclusiones en esta área de investigación.

Por otra parte también se buscaba facilitar el uso de estas técnicas en una aplicación real.
De lo cual surge Alz Care, una pequeña aplicación que integra un sistema DL, en este caso el
previamente obtenido en el desarrollo experimental, pero que puede ser escalable a otros modelos.
Esta aplicación permite al usuario conocer el grado de deterioro cognitivo de una RMI ,y la visualización
de los planos incluidos en la misma, de una manera sencilla y sin tener que instalar ninguna herramienta de
visualización o programa externo.

Pudiendo resultar de gran utilidad en la labor de los médicos o personas de entornos clínicos que tengan este tipo de
necesidades.


\section{Trabajos Futuros}\label{sec:trabajos-futuros}
Con este TFG se obtiene respuesta a las preguntas iniciales, pero con un rendimiento que podría mejorarse con la
utilización de un mayor conjunto de datos.
Como podría ser combinando los biomarcadores de múltiples bases de datos o probando otras técnicas o arquitecturas.

En cuánto a la aplicación web, se ha considerado como un mínimo producto viable.
Sería interesante ampliarla y para ello se establecen tres ideas o puntos de partida que no ha sido posible incluir en
este proyecto por la extensión del mismo.
Las tres ideas que se proponen son:
\begin{itemize}
    \item Ampliar la lógica para que se utilice como aplicación de gestión de paciente de EA en la que se incluya un
    modelo que clasifique más clases y, por lo tanto, pueda servir como seguimiento de la enfermedad.
    \item Agrandar el diagrama arquitectónico de la solución y formar una aplicación que integre más enfermedades y más
    sistemas de clasificación.
    \item Incluir un visualizador completo de RMI en línea.
\end{itemize}
