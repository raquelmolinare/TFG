\chapter{Análisis y Desarrollo Experimental}\label{ch:analisis-y-desarrollo-experimental}
En este capítulo se recoge el estudio de técnicas del DL para el diagnóstico del Alzheimer, pasando por las 
herramientas y los datos utilizados hasta la experimentación y el análisis de los resultados obtenidos.

\section{Presentación del problema a resolver}\label{sec:presentacion-del-problema-a-resolver}
Tras la investigación del estado actual de la literatura en este ámbito, se recopilan los siguientes puntos sobre el DL
para clasificación de la EA:

\begin{itemize}
    \item El Transfer Learning es la mejor técnica de entrenamiento.
    \item Los datasets balanceados ofrecen mejores resultados, y más fiables, en la clasificación de imágenes,
    pero la limitación de disponibilidad de biomarcadores del seguimiento de la EA puede suponer un obstáculo.
    \item La prueba más frecuente para el diagnóstico de la enfermedad es la MRI.
    \item No se clarifica qué plano cerebral (axial, coronal o sagital) es mejor utilizar y bajo qué diferencias de
    rendimiento.
    \item No se realizan comparativas de rendimiento entre distinto número de clases. \\
\end{itemize}

Se propone desarrollar un sistema de aprendizaje profundo a partir de MRI con el cual:

\begin{itemize}
    \item Realizar un análisis sobre qué plano cerebral ofrece mejores resultados en el diagnóstico de la EA, ya que se
    tiene en cuenta que las redes neuronales de 3 dimensiones requieren de una capacidad computacional muy alta.
    \item Realizar una comparativa del rendimiento de la clasificación entre 2 y 3 clases: Cognitivamente normal frente 
    a Alzheimer y cognitivamente normal frente a deterioro cognitivo leve frente Alzheimer.
\end{itemize}

\section{Conjunto de datos empleado}\label{sec:conjunto-de-datos-empleado}

\section{Herramientas utilizadas}\label{sec:herramientas-utilizadas}
Para realizar este estudio es necesario utilizar una serie de herramientas que faciliten y hagan posible la 
experimentación.
Hay que tener en cuenta dos grupos de herramientas: Las herramientas para trabajar con biomarcadores de tipo MRI y las 
herramientas para trabajar con técnicas de DL.

\subsection{Herramientas para trabajar con biomarcadores MRI de tipo NIfTI}
\label{subsec:herramientas-para-trabajar-con-biomarcadores-mri-de-tipo-nifti}
Para poder realizar el sistema de clasificación es necesario realizar un procesamiento de los archivos de MRI, de 
manera que se obtengan cortes de 2 dimensiones a partir del archivo en 3 dimensiones.
Para conseguirlo se ha hecho uso de la librería para python \textbf{NiBabel}, que permite la lectura y escritura  de 
archivos de neuroimagen, en nuestro caso del formato \textit{NIfTI}.
Por lo que mediante esta librería se puede cargar una MRI, extraer el corte o slice deseado y guardarlo.

Además, para la visualización de este tipo de archivos se requiere de software de visualización de imágenes específico.
Hay múltiples herramientas disponibles para ello según la finalidad que se quiera conseguir.
Para visualizar archivos en formato \textit{NIfTI} se ha hecho uso del software \textbf{MRIcron}, que ha requerido 
instalación y con el que se ha podido realizar una visualización de múltiples capas de la MRI. También se ha hecho uso 
del recurso \textbf{IDA} online del que se obtienen los datos de la base de datos \textit{ADNI} y del que se detalla más
información en el el apartado~\ref{ch:aped.a}.
Este recurso integra un visualizador de neuroimágenes.


\subsection{Herramientas para trabajar con técnicas de Deep Learning}
\label{subsec:herramientas-para-trabajar-con-tecnicas-de-deep-learning}
Como bien se muestra en el apartado~\ref{sec:datos-y-herramientas-estado-del-arte}, existe un amplio abanico de
posibilidades en cuanto a herramientas que permiten la experimentación de técnicas de DL.

De entre todas ellas se ha optado por utilizar \textbf{TensorFlow} y \textbf{Keras}, no solo por ser las más utilizadas
en la literatura, sino también porque son las que ofrecen una mejor documentación para su uso.

Para trabajar con imágenes 2D y por lo tanto con vectores y matrices se hace uso de la biblioteca de \textit{Python}
\textbf{NumPy}.

Para visualizar, tanto las imágenes de los planos cerebrales obtenidos como para la representación de las gráficas de
resultados se usa \textbf{Matplotlib}.

Por lo tanto el lenguaje de programación con el que se va a trabajar en este TFG y en concreto en el desarrollo
experimental es \textbf{Python}.

\section{Procesamiento de biomarcadores}

\section{Estrategias de entrenamiento}
\subsection{Transfer Learning}
\subsection{Data Augmentation}

\section{Arquitectura del modelo profundo utilizado}

\section{Experimentos}
\subsection{Comparativa del rendimiento de los planos cerebrales}

\section{Conclusión}
